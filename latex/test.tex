\documentclass[a4paper,12pt,twocolumn]{book}

\title{\textbf{EX VINUM}}
\author{by tsbohc}
\date{}

\usepackage{CormorantGaramond}

\usepackage{tabularx} % page width
\usepackage{calc}
\usepackage{booktabs}
\usepackage{array}
\usepackage[table]{xcolor} % table row colors
\usepackage{hyperref} % links
\usepackage{blindtext} % lorem ipsum
\usepackage[utf8]{inputenc}
\usepackage[russian]{babel}
\usepackage{titlesec} % titles customization
\usepackage{epigraph}
\usepackage{lettrine} % big letters in the beginning of the chapter
\usepackage{filecontents} % define file in text and reuse
%\usepackage{indentfirst}
\usepackage{fancyhdr} % page numbers
\usepackage{tocbasic}

%\raggedbottom

% set page numbers at the bottom
\pagestyle{fancy}
\fancyhf{}
\renewcommand{\headrulewidth}{0pt}
\cfoot{\thepage}

% change space between columns
\setlength{\columnsep}{18pt}

% remove epigraph line
\setlength\epigraphrule{0pt}

% just table things
\definecolor{lightgray}{gray}{0.95}
\newcolumntype{Y}{>{\centering\arraybackslash}X}
\setlength{\extrarowheight}{2pt}

% toc chapters
\setcounter{tocdepth}{1}

\addto\captionsrussian{% Replace "english" with the language you use
  \renewcommand{\contentsname}%
    {Table Of Contents}%
}

\DeclareTOCStyleEntry[
linefill=\bfseries\TOCLineLeaderFill,beforeskip=2pt
]{tocline}{chapter}
\newcommand\chapterprefixintoc[1]{\MakeUppercase{\chaptername}~#1~}

% titles
\titlespacing*{\chapter}{0pt}{12pt plus 4pt minus 4pt}{8pt plus 1pt minus 1pt}
\titlespacing*{\section}{0pt}{12pt plus 4pt minus 4pt}{10pt plus 1pt minus 1pt}
\titlespacing*{\subsection}{0pt}{8pt plus 2pt minus 2pt}{4pt plus 0pt minus 0pt}

\titleclass{\chapter}{straight}

\titleformat{\chapter}
  {\large\bfseries\scshape} % format
  {}                % label
  {0pt}             % sep
  {\LARGE\filcenter} % before-code

\titleformat{\section}
  {\large\bfseries\scshape} % format
  {}                % label
  {0pt}             % sep
  {\Large}          % before-code
  [{\titlerule[0.1pt]}]

\titleformat{\subsection}
  {\large\bfseries\scshape} % format
  {}                % label
  {0pt}             % sep
  {\large}          % before-code

\setlength{\DefaultNindent}{0pt}
\setlength{\DefaultFindent}{6pt}

\begin{document}

\maketitle
\tableofcontents

\chapter{Introduction}
\epigraph{\emph{Да раскинутся границы твоего аутизма до дальних цветочных полей и дымкой затянутых горных вершин.}}{--- бухой волшебник}

\lettrine{E}{x vinum} -- настольная ролевая игра, в сердце которой лежат любовь к рпг и слабоумие, границы между которыми были размыты вином. Так она и получила свое название.

Игра делает акцент на ролевом аспекте, упрощая, но при этом не отказываясь от признанных систем.

Мир игры -- классический фэнтези в стиле D\&D, не претендующий на строгий реализм.

\section{Gameplay}
Dungeon Master описывает ситуацию. Детали окружения, персонажей. Однако, далеко не вся информация доступна игрокам сразу.

Игроки планируют свои действия, задают вопросы. Приняв решения, оглашают их, а затем бросают кости. Успешность действий зависит от сложности их выполнения и аттрибутов соотвествующих персонажей.

DM описывает результаты действий игроков.

\section{Nota Bene}
DM и игроки работают вместе, чтобы создать общий нарратив, а не играют друг против друга.

Что бы ни было написано в правилах, DM всегда имеет последнее слово.

DM должен поощрять изобретательность игроков их способность вживаться в роли, даже если их идеи не реалистичны. DM должен поощрять способность игроков вживаться в роли.

Главная цель не победить, а хорошо провести время в кругу друзей.

\section{Your Character}

Кем будешь ты? Добродушным дворфом с раскатистым смехом, знающим лучшие таверны в городе? Или тифлингом, чьи руки в крови, потому что единственным выбором была смерть? В этой главе мы разберем общие аспекты создания персонажа.

\subsection{Race}

Выбор расы влияет не только на физический облик и характеристики, но и ваше положение в мире, а также отношения с другими рассами. Кроме этого, ваша история вытекает из происхождения.

\subsection{Age}

Некоторые расы, например дворфы и эльфы, живут на несколько сотен лет дольше чем люди. В результате чего обладают другой перспективой на мир.

\subsection{Sex}

Пол обладает большим влиянием на ваш характер и прошлое. Не все женщины, покинув родной дом тут же становятся рыцарями. Рыцарями становятся те женщины, чьим сердцам дорого то, что они хотят защитить.

\subsection{Class}

Каждый искатель приключений относится к тому или иному классу. Класс определяет, какими навыками вы обладаете и какую роль вам предстоит выполнять в компании. А также, владаеете ли вы магией или хорошо управляетесь с оружием.

\subsection{Personality}

Личность составляют идеи которые вами движат и недостатки которыми обладаете. Ваши сильные и слабые стороны, ваш характер, и то, как вы ведете себя с другими людьми.

%Потух ли огонь ваших глаз за годы скитаний, или вы только в начале своего пути?

\subsection{Backstory}

Какой была ваша жизнь до начала кампании? Может быть, вас отправили на важное задание, или вовсе изгнали? Как вы преобрели свои навыки и стали своим классом? Что вы хотите защитить или чем желаете овладеть? Есть ли у вас возлюбленный или возлюбленная? Потеряли ли вы старого друга или преследуете врага?

\subsection{Appearance}

Все, что связано с внешним обликом. Телесложение, цвет кожи, волос, глаз. Шрамы, полученные в бою и татуировки. Детали одежды и украшения, талисманы.

\chapter{Attributes}

\lettrine{П}{}ерсонажи ex vinum обладают всего тремя аттрибутами, каждый из которых имеет второстепенный эффект.

Аттрибуты говорят о навыках персонажа и выражаются численными значениями.

\subsection{Strength}

Определяет физическую силу и выносливость -- \textbf{Constitution}. Имеет большое влияние на общий запас здоровья.

\subsection{Dexterity}

Определяет ловкость и скорость реакции, а также внимательность -- \textbf{Perception}. Слабо влияет на запас здоровья.

\subsection{Intelligence}

Определяет ум и сообразительность, фоновые знания, а также уменее вести разговор -- \textbf{Charisma}. Не влияет на запас здоровья.

\chapter{Races}
\lettrine{Н}{}иже перечислены разумные, игровые расы ex vinum. Полный список существ далеко не закачивается этим разделом.

\section{Human}
Ну, вы поняли. Люди преуспевают во многом, но реже добиваются невероятных результатов. Являются самой многочисленной расой, включающей в себя огромное количество разных культур.

\smallskip
+1 любому аттрибуту на выбор.

\section{Dwarf}

Воспитанные грубой средой обитания и тяжелой работой, Дворфы ценят свое мастерство. Чаще всего они становятся ремеслeнниками или шахтерами. Любят провести вечер пересчитывая свое золото за парой-тройкой-десятком кружек эля.

Невысокого роста. Тучные, мускулистые, с бурым или темным цветом волос, грубоватым голосом и громким смехом.

\smallskip
+2 strength, -1 dexterity

\section{Elf}

Изящные и благоразумные, Эльфы тесно связаны с магией этого мира. Часто становятся писателями или художниками. Большую часть времени проводят наедине с природой и книгами.

Ростом чуть ниже человека. Стройные, светловолосые, часто с голубым или зеленым оттенком глаз. Голоса мелодичны и благозвучны.

\smallskip
+2 intelligence, -1 strength

\section{Tiefling}

Тифлинги несут за собой бремя своей родословной -- они потомки людей и дьявола. Их полу-демоническое обличие не пробуждают в людях теплых чувств, лишь только косые взгляды. Тифлинги, оказавшиеся в этом мире не по своей воли, делают все, чтобы выжить.

Ростом с человека. Цвета кожи те же, что и у людей, но могут принимать и красноватые оттенки. Длииинный хвост. На голове большие рога. Туманные глаза, темные волосы.

\smallskip
+2 dexterity, -1 intelligence

%\section{Duergar}
%
%\smallskip
%+1 strength, +1 dexterity, -1 intelligence

\section{Drow}

Темные Эльфы произошли от дверней подрасы Эльфов, изгнанных обитать в глубинах земли. Те, что выбираются на поверхность, нередко идут по пути зла, но встречаются и исключения. По большей части становятся путешественниками или наемниками.

Чуть меньше и стройнее своих собратьев. Обладают темным цветом кожи, практически белым цветом волос и бледными глазами с оттенком.

\smallskip
-1 strength, +1 dexterity, +1 intelligence

\section{Aasimar}

Азимары подобны Тифлингам в своем происхождении, но в их жилах течет божественная кровь.  Азимары от природы расположены к благотелям, а в своих снах ведут разговоры со своим божеством. Однако, предпочитают не раскрывать свою родословную.

Выше человека. В цветах волосы и кожи встречаются как человеческие, так и более металлические оттенки. Другими рассамы описываются как люди невероятной красоты. 


%\noindent\fbox{
%\parbox{\columnwidth}{
%The quick brown fox jumps right over the lazy dog. the quick brown fox jumps right over the lazy dog. the quick brown fox jumps right over the lazy dog. the quick brown fox jumps right over the lazy dog. the quick brown fox jumps right over the lazy dog. the quick brown fox jumps right over the lazy dog. the quick brown fox jumps right over the lazy dog. the quick brown fox jumps right over the lazy dog.
%}
%}

\smallskip
+1 strength, -1 dexterity, +1 intelligence

%\chapter{Rules}
%
%\lettrine{В}{}центре ролевой системы игры -- d12. Такой выбор обусловлен желанием добавить динамики более размеренному тону d20.
%
%Провалы и успехи пропорциональны сложности выполняемого действия.

\chapter{Character Creation}

\lettrine{С}{}ейчас хорошее время, чтобы отложить эту книгу и подумать за кого вы хотите играть. Встретившись за столом, вам предстоит рассказать о своем персонаже и познакомиться с другими участниками группы.

Один совет -- попробуйте себя в роли, не совсем похожей на вас -- но такой, которую вам интересно было бы исполнять. И обязательно все запишите.

Когда вы будете готовы, мы перейдем к непосредственно к числам, и тому, как они будут определяться за столом.

\section{Attributes}

\subsection{Score Rolling}

Чтобы определить свои базовые атрибуты, игрок бросает 3d4 и складывает полученные значения. Затем повторяет этот процесс три раза. Итоговые роллы игрок распределяет на свое усмотрение.

Перед началом игры DM может предложить свой метод определения атрибутов.

\subsection{Best Backstory Award}

Самому интересному персонажу по мнению большинства DM предоставит +1 к любомy атрибуту.

\subsection{Racial Bonuses}

На следующем этапе, к базовым аттрибутам добавляются расовые бонусы. Здесь они собраны в одну таблицу для удобства.

Люди добавляют +1 к любому выбранному аттрибуту.

\begin{table}[htb]
\rowcolors{1}{}{lightgray}
\centering
\setlength\tabcolsep{0pt}
\begin{tabularx}{\linewidth}{@{}Y|YYY@{}}
  \emph{race} & \emph{str} & \emph{dex} & \emph{int} \\
  \hline
  human     & +? & +? & +? \\
  dwarf     & +2 & -1 &    \\
  elf       & -1 &    & +2 \\
  tiefling  &    & +2 & -1 \\
  %duergar   & +1 & +1 & -1 \\
  drow      & -1 & +1 & +1 \\
  aasimar   & +1 & -1 & +1
\end{tabularx}
\end{table}

\subsection{Max Health}

Максимальный запас здоровья расчитывается по следующей формуле:

\[hp = strength*6 + dexterity*2\]

\subsection{Attribute Modifiers}
Моды используются чтобы подчеркнуть, насколько хорошо или плохо персонаж владеет тем или иным навыком.

Высокие значения аттрибутов пораждают положительные моды, низкие -- отрицательные.

\begin{table}[htb]
\rowcolors{1}{}{lightgray}
\centering
\setlength\tabcolsep{0pt}
\begin{tabularx}{\linewidth}{@{}YY|YY@{}}
  \emph{attribute} & \emph{modifier} & \emph{attribute} & \emph{modifier} \\
  \hline
  1-2 & -3 & 9-10 & +1 \\
  3-4 & -2 & 11-12 & +2 \\
  5-6 & -1 & 13-14 & +3 \\
  7-8 & -0 & 15-16 & +4
\end{tabularx}
\end{table}

Расчитать свои моды можно по таблице выше или обратившись к формуле:

\[attr\ mod = floor((attr - 6.5)/2)\]

\chapter{Rules}

\lettrine{В}{}этом разделе мы поговорим о тонкостях проведения боя, отдыхе и других игровых механиках.

\section{Ability Checks}

В случае, когда действие игрока имеет шанс на провал, DM просит его сделать ролл на проверку аттрибута.

Успешность определяется сложением ролла d12 и мода, указанного DM.

Для успешного выполнения действия, итоговый ролл должен быть равен или превосходить сложность установленную DM. Сложность при этом не разглашается.

Другими словами, действие успешно если:

\[d12 + attr\ modifier \geq difficulty \]

\section{Free Movement}

Вне боя игроки передвигаются свободно и взаимодействуют с окружением без какого либо установленного порядка.

\section{Combat}

Бой происходит по раундами. За один раунд каждый участник боя включая игроков и других персонажей ходит один раз.

\subsection{Initiative}

Если игроки устраивают засаду, у них есть возможность установить порядок ходов между собой.

В противном случае, порядок определяется роллами \(d12 + dex\ mod\) от большего к меньшему.

\subsection{Vision}

Персонажи могу взаимодействовать только с теми объектами, которые видят или чувствуют другим образом. Стены, дым, темнота и т.д. препятствуют зрению.

\subsection{Rolling}

Успешность каждого действия в бою определяется одним роллом d12. Урон, наносимый атакой или заклинанием почти всегда равен данному роллу. Исключения составляют критические провалы, когда все идет не по плану, и критические успехи. В случае с последними, урон удваивается.

\subsection{Opportunity Attacks}

Если персонаж, находясь в ближем бою с другим персонажем делает попытку сбежать, по нему проходит атака возможности. У данной ситуации есть исключения: критический успех побега, использование умения.

% 5 damage

\section{Resting}

На протяжение кампании у игроков будет возможность разбить лагерь и отдохнуть. Качество отдыха определяется действиями персонажей. Отдых восстанавливает игрокам здоровье.

Для хорошего отдыха необходимо хорошо поесть, обустроить лагерь и рассказать пару историй у костра.

\chapter{Bestiary}


%\chapter{практические примеры}
%
%\hypertarget{abilitycheckexample}{\section{ability check}}
%
%Персонаж Josh довольно слабый, у него всего 4 strength:
%
%\*
%
%\(str\ b = floor((4 - 7)/2)\)
%
%\(str\ b = floor(-3/2)\)
%
%\(str\ b = floor(-1.5)\)
%
%\*
%
%\emph{floor} говорит о том, что полученное значение округляется вниз. Таким образом:
%
%\[strength\ bonus = -2\]
%
%Представим ситуацию, что Josh хочет пробить плечем дверь. DM просит игрока сделать ability check на strength. Игрок берет d12 и выбрасывает 7. Обращаясь к формуле выше, к полученному роллу он прибавляет свой бонус:
%
%\[7 + (-2) = 5\]
%
%Допустим, что дверь совсем новая, и прочно укреплена. DM выставил сложность действия 6 (среднея). Итоговый ролл \emph{y} меньше данного числа, и дверь не поддается. 
%
%В противном случае, представим что дверь ветхая и давно прогнила. DM выставил сложность 3 (низкая) и Josh пробивает ее не смотря на свою невысокую силу.
%







\end{document}
