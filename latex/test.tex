\documentclass[a5paper,11pt]{book}

% {{{ latex things

\title{\textbf{EX VINUM}}
\author{by tsbohc}
\date{}

\usepackage{longtable}
\usepackage{multicol}
\usepackage[linewidth=0.1pt]{mdframed}
\usepackage{CormorantGaramond}
\usepackage{tabularx} % page width
\usepackage{calc}
\usepackage{booktabs}
\usepackage[margin=0.6in, bottom=0.7in]{geometry}
\usepackage{array}
\usepackage[table]{xcolor} % table row colors
\usepackage{hyperref} % links
\usepackage{blindtext} % lorem ipsum
\usepackage[utf8]{inputenc}
\usepackage[russian]{babel}
\usepackage{titlesec} % titles customization
\usepackage{epigraph}
%\usepackage{graphicx}
%\usepackage{background}
\usepackage{lettrine} % big letters in the beginning of the chapter
%\usepackage{indentfirst}
\usepackage{fancyhdr} % page numbers
\usepackage{tocbasic}

\newmdenv[
innerbottommargin=11pt, innertopmargin=\topskip, font=\itshape
]{boxed}

%\hyphenpenalty=100000

% messing with layout
%\raggedbottom
%\widowpenalty=1000
%\clubpenalty=1000

%\backgroundsetup{scale=0.75,angle=0,contents=\includegraphics{paper.jpg}}

% set page numbers at the bottom
\pagestyle{fancy}
\fancyhf{}
\renewcommand{\headrulewidth}{0pt}
\cfoot{\thepage}

% change space between columns
\setlength{\columnsep}{18pt}

% remove epigraph line 
\setlength\epigraphrule{0pt}

% just table things
\definecolor{lightgray}{gray}{0.95}
\definecolor{white}{gray}{1}
\newcolumntype{Y}{>{\centering\arraybackslash}X}
\newcolumntype{Z}{>{\ \ \ }l}
\newcolumntype{P}[1]{>{\centering\arraybackslash}p{#1}}
\setlength{\extrarowheight}{2pt}

% toc chapters
\setcounter{tocdepth}{1}

\addto\captionsrussian{% Replace "english" with the language you use
  \renewcommand{\contentsname}%
    {Table Of Contents}%
}

\DeclareTOCStyleEntry[
linefill=\bfseries\TOCLineLeaderFill,beforeskip=2pt
]{tocline}{chapter}
\newcommand\chapterprefixintoc[1]{\MakeUppercase{\chaptername}~#1~}

% titles
\titlespacing*{\chapter}{0pt}{12pt plus 4pt minus 4pt}{8pt plus 1pt minus 1pt}
\titlespacing*{\section}{0pt}{12pt plus 4pt minus 4pt}{10pt plus 1pt minus 1pt}
\titlespacing*{\subsection}{0pt}{8pt plus 2pt minus 2pt}{4pt plus 0pt minus 0pt}

\titleclass{\chapter}{straight}

%\definecolor{reddish}{RGB}{76,28,12}
\definecolor{reddish}{RGB}{0,0,0}

\titleformat{\chapter}
  {\color{reddish}\huge\bfseries\scshape} % format
  {}                % label
  {0pt}             % sep
  {} % before-code
  [{\titlerule[0.1pt]}]

\titleformat{\section}
  {\color{reddish}\large\bfseries\scshape} % format
  {}                % label
  {0pt}             % sep
  {\Large}          % before-code
  [{\titlerule[0.1pt]}]

\titleformat{\subsection}
  {\color{reddish}\large\bfseries\scshape} % format
  {}                % label
  {0pt}             % sep
  {\large}          % before-code

\setlength{\DefaultNindent}{0pt}
\setlength{\DefaultFindent}{6pt}

\begin{document}

\maketitle
\thispagestyle{empty}
\enlargethispage{\baselineskip}
\tableofcontents
\pagebreak

% }}}

% {{{ intro
\chapter{Introduction}
\begin{multicols}{2}
\epigraph{\emph{Да раскинутся границы твоего аутизма до дальних цветочных полей и дымкой затянутых горных вершин.}}{--- бухой волшебник}

\lettrine{E}{x vinum} -- настольная ролевая игра, в сердце которой лежат любовь к рпг и слабоумие, границы между которыми были размыты вином. Так она и получила свое название.

Игра делает акцент на ролевом аспекте, упрощая, но при этом не отказываясь от признанных систем.

Мир игры -- классический фэнтези в стиле D\&D, не претендующий на строгий реализм.

\section{Gameplay}
Dungeon Master описывает ситуацию. Детали окружения, персонажей. Однако, далеко не вся информация доступна игрокам сразу.

Игроки описывают, что они хотят сделать. Во время боя это происходит по порядку очередности ходов. Вне боя, игроки могут действовать свободно -- один может взламывать сундук, а другой стоять на стороже.

Иногда решить проблему легко, например открыть дверь. В таком случае DM может просто сказать, что дверь открылась, и описать, что лежит за ней. В других случаях, успешность действий игроков определяется бросками костей.

DM описывает результаты действий игроков, и на этом цикл повторяется.

Стоит помнить, что это не компьютерная игра. Действия участников ограничиваются лишь их воображением и фэнтезийной версией законов физики, а минмаксинг может обернуться плачевно.

\section{Nota Bene}
DM и игроки работают вместе, чтобы создать общий нарратив, а не играют друг против друга.

Что бы ни было написано в правилах, DM всегда имеет последнее слово.

DM должен поощрять изобретательность игроков их способность вживаться в роли, даже если их идеи не реалистичны.

Главная цель не победить, а хорошо провести время в кругу друзей.

% }}}

% {{{ races 
\end{multicols}
\chapter{Races}
\begin{multicols}{2}

\lettrine{Н}{}иже перечислены разумные, игровые расы ex vinum. Полный список существ далеко не закачивается этим разделом.

\section{Human}
Ну, вы поняли. Люди преуспевают во многом, но реже добиваются невероятных результатов. Являются самой многочисленной расой, включающей в себя огромное количество разных культур.

\section{Dwarf}

Воспитанные грубой средой обитания и тяжелой работой, Дворфы ценят свое мастерство. Чаще всего они становятся ремеслeнниками или шахтерами. Любят провести вечер пересчитывая свое золото за парой-тройкой-десятком кружек эля.

Невысокого роста. Тучные, мускулистые, с бурым или темным цветом волос, грубоватым голосом и громким смехом.

%Сильные и выносливые, но слегка неповоротливые.

\section{Elf}

Изящные и благоразумные, Эльфы тесно связаны с магией этого мира. Часто становятся писателями или художниками. Большую часть времени проводят наедине с природой и книгами.

Ростом чуть ниже человека. Стройные, светловолосые, часто с голубым или зеленым оттенком глаз. Голоса мелодичны и благозвучны.

%Умные и харизматичные, но не обладают большой физической силой.

\section{Tiefling}

Тифлинги несут за собой бремя своей родословной -- они потомки людей и дьявола. Их полу-демоническое обличие не пробуждают в людях теплых чувств, лишь только косые взгляды. Тифлинги, оказавшиеся в этом мире не по своей воли, делают все, чтобы выжить.

Ростом с человека. Цвета кожи те же, что и у людей, но могут принимать и красноватые оттенки. Длииинный хвост. На голове большие рога. Туманные глаза, темные волосы.

%\smallskip
%+2 dexterity, -1 intelligence

%\section{Duergar}
%
%\smallskip
%+1 strength, +1 dexterity, -1 intelligence

\section{Drow}

Темные Эльфы произошли от дверней подрасы Эльфов, изгнанных обитать в глубинах земли. Те, что выбираются на поверхность, нередко идут по пути зла, но встречаются и исключения. По большей части становятся путешественниками или наемниками.

Чуть меньше и стройнее своих собратьев. Обладают темным цветом кожи, практически белым цветом волос и бледными глазами с оттенком.

%\smallskip
%-1 strength, +1 dexterity, +1 intelligence

\section{Aasimar}

Азимары подобны Тифлингам в своем происхождении, но в их жилах течет божественная кровь.  Азимары от природы расположены к благотелям, а в своих снах ведут разговоры со своим божеством. Однако, предпочитают не раскрывать свою родословную.

Выше человека. В цветах волосы и кожи встречаются как человеческие, так и более металлические оттенки. Другими рассамы описываются как люди невероятной красоты. 


%\smallskip
%+1 strength, -1 dexterity, +1 intelligence

% }}}

% {{{ roleplaying

\end{multicols}
\chapter{Roleplaying}
\begin{multicols}{2}
\lettrine{К}{}ем будешь ты? Добродушным дворфом с раскатистым смехом, знающим лучшие таверны в городе? Или тифлингом, чьи руки в крови, потому что единственным выбором была смерть?

\section{Race}
Выбор расы влияет не только на физический облик и характеристики, но и ваше положение в мире, а также отношения с другими рассами. Кроме этого, ваша история вытекает из происхождения.

\section{Age}
Некоторые расы, например дворфы и эльфы, живут на несколько сотен лет дольше чем люди. В результате чего имеют другую перспективу на мир.

\section{Sex}
Пол обладает большим влиянием на ваш характер и прошлое. Не все женщины покинув родной дом тут же становятся рыцарями. Рыцарями становятся те женщины, чьим сердцам дорого то, что они хотят защитить.

\section{Class}
Каждый искатель приключений обладает той или иной профессией. Профессия определяет, владаеете ли вы магией или хорошо управляетесь с оружием, и какую роль вам предстоит выполнять в кампании.

\section{Personality}
Личность составляют идеи которые вами движат и недостатки которыми обладаете. Ваши сильные и слабые стороны, ваш характер, и то, как вы ведете себя с другими людьми.

\section{Backstory}
Какой была ваша жизнь до начала кампании? Может быть, вас отправили на важное задание, или вовсе изгнали? Как вы преобрели свои навыки и стали своим классом? Что вы хотите защитить или чем желаете овладеть? Есть ли у вас возлюбленный или возлюбленная? Потеряли ли вы старого друга или преследуете врага?

\section{Appearance}
Все, что связано с внешним обликом. Телесложение, цвет кожи, волос, глаз. Шрамы, полученные в бою и татуировки. Детали одежды и украшения, талисманы.

% }}}

% {{{ character creation

\end{multicols}
\chapter{Character Creation}
\begin{multicols}{2}

\lettrine{С}{}ейчас хорошее время, чтобы отложить эту книгу и подумать за кого вы хотите играть. Встретившись за столом, вам предстоит рассказать о своем персонаже и познакомиться с другими участниками группы.

Один совет -- попробуйте себя в роли, не совсем похожей на вас -- но такой, которую вам интересно было бы исполнять.

\section{Dice}

Броски костей обозначаются двумя числами через букву d. Например, 3d6 означает три броска кости с 6 сторонами. В случае отсутствия других указаний, подразумевается сумма указанных бросков.

В случае с 2d8+3 к сумме двух бросков d8 прибавляется 3.

В основе ролевой системы ex vinum лежит d12. Это обусловлено желанием добавить динамики традиционной системе d20.

\section{Attributes}
Атрибуты выражают физические и умственные способности персонажа. Определяются численными значениями, в среднем от 3 до 12.

Например, персонаж c 3 интеллекта не способен читать, крайне рассеян и может заговариться.

\subsection{Strength}
Определяет физическую силу и выносливость. Имеет большое влияние на общий запас здоровья.

\subsection{Dexterity}
Определяет ловкость и скорость реакции, а также внимательность. Слабо влияет на запас здоровья.

\subsection{Intelligence}
Определяет ум и сообразительность, фоновые знания, а также уменее вести разговор. Не влияет на запас здоровья.

\section{Rolling Attributes}
Чтобы определить свои базовые атрибуты, игрок бросает 3d4 и складывает полученные значения. Затем повторяет этот процесс еще два раза. Полученные три ролла игрок распределяет на свое усмотрение.

Перед началом игры DM может предложить свой метод определения атрибутов.

\subsection{Best Backstory Award}
Самому интересному персонажу по мнению большинства DM предоставит +1 к выбранному им атрибуту.

\subsection{Racial Bonuses}
На данном этапе к базовым атрибутам применяются расовые бонусы.

\smallskip
\noindent
\rowcolors{1}{}{lightgray}
\setlength\tabcolsep{0pt}
\begin{tabularx}{\linewidth}{@{}ZYYY@{}}
  \textbf{race} & \textbf{str} & \textbf{dex} & \textbf{int} \\
  \hline
  dwarf     & +2 & -1 &    \\
  elf       & -1 &    & +2 \\
  tiefling  &    & +2 & -1 \\
  %duergar   & +1 & +1 & -1 \\
  drow      & -1 & +1 & +1 \\
  aasimar   & +1 & -1 & +1
\end{tabularx}
\smallskip

Люди прибавляют +1 к любому выбранному атрибуту.

\subsection{Health}

Максимальный запас здоровья расчитывается по следующей формуле: \textbf{6 × strength + 2 × dexterity}.

\section{Attribute Modifiers}
Моды используются чтобы подчеркнуть, насколько хорошо или плохо персонаж владеет тем или иным навыком, и участвуют в подсчете роллов.

Высокие значения атрибутов пораждают положительные моды, низкие -- отрицательные.

Расчитать свои моды можно обратившись к таблице:

\smallskip
\noindent
\rowcolors{1}{}{lightgray}
\setlength\tabcolsep{0pt}
\begin{tabularx}{\linewidth}{@{}YYYY@{}}
  \textbf{attr} & \textbf{mod} & \textbf{attr} & \textbf{mod} \\
  \hline
  1-2 & -3 & 9-10 & +1 \\
  3-4 & -2 & 11-12 & +2 \\
  5-6 & -1 & 13-14 & +3 \\
  7-8 & -0 & 15-16 & +4
\end{tabularx}

% }}}

% {{{ character sheet 

\end{multicols}
\chapter{Character Example}

\setlength{\arrayrulewidth}{0.1pt}

\noindent
\setlength\tabcolsep{0pt}
\def\arraystretch{1.3}
\begin{tabularx}{\linewidth}{@{}YYYYYYYYYYYYY@{}}

  \multicolumn{13}{c}{\large\sc\textbf{Name:}} \\ 
  \multicolumn{13}{c}{\Large\emph{Ailon Dawnguard}} \\ \cline{4-10}

  \addlinespace[0.5cm]

  \multicolumn{6}{c}{\large\sc\textbf{Attributes:}} &
  \multicolumn{1}{c}{} &
  \multicolumn{6}{c}{\large\sc\textbf{Character:}}
  \\ \cline{1-6}

  \multicolumn{2}{c|}{\small\sc\textbf{Str}} &
  \multicolumn{2}{c|}{\small\sc\textbf{Dex}} &
  \multicolumn{2}{c}{\small\sc\textbf{Int}} &
  \multicolumn{1}{c}{} &
  \multicolumn{6}{Z}{\emph{Полу-эльф Волшебник, 24, ж.}}
  \\ \cline{8-13}

  \multicolumn{2}{c|}{\LARGE{+0}} &
  \multicolumn{2}{c|}{\LARGE{+1}} &
  \multicolumn{2}{c}{\LARGE{+2}} &
  \multicolumn{1}{c}{} &
  \multicolumn{6}{Z}{\emph{Сбежала из монастыря чтобы}}
  \\ \cline{1-6} \cline{8-13}

  \multicolumn{2}{c|}{\large{7}} & 
  \multicolumn{2}{c|}{\large{9}} &
  \multicolumn{2}{c}{\large{11}} &
  \multicolumn{1}{c}{} &
  \multicolumn{6}{Z}{\emph{увидеть мир таким, какой он}}
  \\ \cline{8-13}

  \multicolumn{6}{c}{\large\sc\textbf{Stats:}}  &
  \multicolumn{1}{c}{} &
  \multicolumn{6}{Z}{\emph{есть. Хочет стать лучше, но}}
  \\ \cline{1-6} \cline{8-13}

  \multicolumn{3}{c|}{\small\sc\textbf{Max Hp}} &
  \multicolumn{3}{c}{\small\sc\textbf{Gold}} &
  \multicolumn{1}{c}{} &
  \multicolumn{6}{Z}{\emph{боится потерять себя. Веснушки,}}
  \\ \cline{8-13}

  \multicolumn{3}{c|}{\LARGE{46}} &
  \multicolumn{3}{c}{\LARGE{75}} &
  \multicolumn{1}{c}{} &
  \multicolumn{6}{Z}{\emph{русые волосы, невысокий рост.}}
  \\ \cline{1-6} \cline{8-13}
  &  &  &  &  &  &  &  &  &  &  &  & \\
\end{tabularx}

\begin{multicols}{2}

Выше живой пример заполненного листа персонажа, на основе моего протагониста \emph{Baldur's Gate 2}.

Игрокам будут выданы дополнительные листки для заметок -- на них можно будет вести учет своего золота, здоровья, состояний и предметов в инвентаре.

Стоит отметить, что правая колонка -- краткое описание персонажа, на которое я бы опирался, рассказывая его полную историю.

% }}}

% {{{ rules

\end{multicols}
\chapter{Rules}
\begin{multicols}{2}

\section{Ability Checks}

\lettrine{В}{}случае, когда действие игрока имеет шанс на провал, DM просит его сделать ролл на проверку аттрибута.

Успешность определяется сложением ролла d12 и мода, указанного DM.

Для успешного выполнения действия, итоговый ролл должен быть равен или превосходить сложность установленную DM. Сложность при этом не разглашается.

\subsection{Advantage}
У игроков может быть преимущество в выполнении того или иного действия. В этом случае, игрок делает два ролла d12 в проверке атрибута и выбирает наибольший.

\section{Out of Combat}
Вне боя игроки передвигаются свободно и взаимодействуют с окружением без какого либо установленного порядка.

\subsection{Assisting}
Один из игроков можем помочь другому в проверке атрибута, предварительно кинув d12. Если ролл больше 6, игрок, которому помогают, получает преимущество.

\section{In Combat}
Бой происходит по раундами. За один раунд каждый участник боя включая игроков и других персонажей ходит один раз.

\subsection{Initiative}
Если игроки устраивают засаду, у них есть возможность установить порядок ходов между собой.

В противном случае, порядок определяется роллами \textbf{d12 + dex mod} от большего к меньшему.

\subsection{Vision}
Персонажи могу взаимодействовать только с теми объектами, которые видят или чувствуют другим образом. Стены, дым, темнота и т.д. препятствуют зрению.

\subsection{Rolling}
Успешность каждого действия в бою определяется одним роллом d12. 

Урон, наносимый атакой или заклинанием равен сумме данного ролла и соответствующего мода. Исключения составляют критические провалы, когда все идет не по плану, и критические успехи. В случае с последними, урон удваивается.

\subsection{Opportunity Attacks}
Если персонаж, находясь в ближем бою с другим персонажем делает попытку сбежать, по нему проходит атака возможности. У данной ситуации есть исключения: критический успех побега, использование умения.

% 5 damage

\section{Resting}
На протяжение кампании у игроков будет возможность разбить лагерь и отдохнуть. Качество отдыха определяется действиями персонажей. Отдых восстанавливает игрокам здоровье.

Для хорошего отдыха необходимо хорошо поесть, обустроить лагерь и рассказать пару историй у костра.

% }}}

% {{{ equipment


\end{multicols}
\pagebreak
\chapter{Equipment}

\noindent
\rowcolors{1}{}{lightgray}
\setlength\tabcolsep{0pt}
\setlength\LTleft{0pt}
\setlength\LTright{0pt}
\begin{longtable}{p{0.4\textwidth}P{0.1\textwidth}p{0.4\textwidth}P{0.1\textwidth}}
  \textbf{\ \ \ name}
  & \textbf{price}
  & \textbf{\ \ \ name}
  & \textbf{price}
  \\
  \hline
  \ \ \ зелье здоровья                  &  25g &
  \ \ \ зелье языка животных            &  10g \\
  \ \ \ зелье ночного зрения            &  10g &
  \ \ \ зелье подводного дыхания        &  10g \\
  \ \ \ зелье паучьих ног               &  25g &
  \ \ \ фляска кислоты                  &  10g \\
  \ \ \ фляска вечной мерзлоты          &  10g &
  \ \ \ фляска дымовой завесы           &   5g \\
  \ \ \ фляска света                    &  10g &
  \ \ \ порошок усыпления               &   5g \\
  \ \ \ горсть пороха                   &   5g &
  \ \ \ палатка                         &  10g \\
  \ \ \ котел                           &   5g &
  \ \ \ огниво                          &   5g \\
  \ \ \ соль и перец                    &   2g &
  \ \ \ бинты                           &  15g \\
  \ \ \ бутылка виски                   &   3g &
  \ \ \ отмычки                         &   5g \\
  \ \ \ крюк                            &   5g &
  \ \ \ железные колючки                &   5g \\
  \ \ \ зеркало                         &  10g &
  \ \ \ факел                           &   3g \\
  \ \ \ веревка                         &   5g &
  \ \ \ мыло                            &   1g \\
  \ \ \ лопата                          &   5g &
  \ \ \ топорик                         &   5g \\
  \ \ \ кирка                           &   5g &
  \ \ \ музыкальный инструмент          &   5g
  \\

\end{longtable}

По поводу наличия неуказанных выше предметов можно спросить у торговца, главное сделать это в будучи в роли.

\begin{multicols}{2}

% }}}

% {{{ print only

\end{multicols}
\pagebreak


\rowcolors{1}{}{}

\noindent
\setlength\tabcolsep{0pt}
\def\arraystretch{1.3}
\begin{tabularx}{\linewidth}{@{}YYYYYYYYYYYYY@{}}

  \multicolumn{13}{c}{\large\sc\textbf{Name:}} \\ 
  \multicolumn{13}{c}{\Large\emph{}} \\ \cline{4-10}

  \addlinespace[0.5cm]

  \multicolumn{6}{c}{\large\sc\textbf{Attributes:}} &
  \multicolumn{1}{c}{} &
  \multicolumn{6}{c}{\large\sc\textbf{Character:}}
  \\ \cline{1-6}

  \multicolumn{2}{c|}{\small\sc\textbf{Str}} &
  \multicolumn{2}{c|}{\small\sc\textbf{Dex}} &
  \multicolumn{2}{c}{\small\sc\textbf{Int}} &
  \multicolumn{1}{c}{} &
  \multicolumn{6}{Z}{\emph{}}
  \\ \cline{8-13}

  \multicolumn{2}{c|}{\LARGE{}} &
  \multicolumn{2}{c|}{\LARGE{}} &
  \multicolumn{2}{c}{\LARGE{}} &
  \multicolumn{1}{c}{} &
  \multicolumn{6}{Z}{\emph{}}
  \\ \cline{1-6} \cline{8-13}

  \multicolumn{2}{c|}{\large{}} & 
  \multicolumn{2}{c|}{\large{}} &
  \multicolumn{2}{c}{\large{}} &
  \multicolumn{1}{c}{} &
  \multicolumn{6}{Z}{\emph{}}
  \\ \cline{8-13}

  \multicolumn{6}{c}{\large\sc\textbf{Stats:}}  &
  \multicolumn{1}{c}{} &
  \multicolumn{6}{Z}{\emph{}}
  \\ \cline{1-6} \cline{8-13}

  \multicolumn{3}{c|}{\small\sc\textbf{Max Hp}} &
  \multicolumn{3}{c}{\small\sc\textbf{Gold}} &
  \multicolumn{1}{c}{} &
  \multicolumn{6}{Z}{\emph{}}
  \\ \cline{8-13}

  \multicolumn{3}{c|}{\LARGE{}} &
  \multicolumn{3}{c}{\LARGE{}} &
  \multicolumn{1}{c}{} &
  \multicolumn{6}{Z}{\emph{}}
  \\ \cline{1-6} \cline{8-13}
  &  &  &  &  &  &  &  &  &  &  &  & \\
\end{tabularx}

\vspace{40pt}

\noindent
\setlength\tabcolsep{0pt}
\def\arraystretch{1.3}
\begin{tabularx}{\linewidth}{@{}YYYYYYYYYYYYY@{}}

  \multicolumn{13}{c}{\large\sc\textbf{Name:}} \\ 
  \multicolumn{13}{c}{\Large\emph{}} \\ \cline{4-10}

  \addlinespace[0.5cm]

  \multicolumn{6}{c}{\large\sc\textbf{Attributes:}} &
  \multicolumn{1}{c}{} &
  \multicolumn{6}{c}{\large\sc\textbf{Character:}}
  \\ \cline{1-6}

  \multicolumn{2}{c|}{\small\sc\textbf{Str}} &
  \multicolumn{2}{c|}{\small\sc\textbf{Dex}} &
  \multicolumn{2}{c}{\small\sc\textbf{Int}} &
  \multicolumn{1}{c}{} &
  \multicolumn{6}{Z}{\emph{}}
  \\ \cline{8-13}

  \multicolumn{2}{c|}{\LARGE{}} &
  \multicolumn{2}{c|}{\LARGE{}} &
  \multicolumn{2}{c}{\LARGE{}} &
  \multicolumn{1}{c}{} &
  \multicolumn{6}{Z}{\emph{}}
  \\ \cline{1-6} \cline{8-13}

  \multicolumn{2}{c|}{\large{}} & 
  \multicolumn{2}{c|}{\large{}} &
  \multicolumn{2}{c}{\large{}} &
  \multicolumn{1}{c}{} &
  \multicolumn{6}{Z}{\emph{}}
  \\ \cline{8-13}

  \multicolumn{6}{c}{\large\sc\textbf{Stats:}}  &
  \multicolumn{1}{c}{} &
  \multicolumn{6}{Z}{\emph{}}
  \\ \cline{1-6} \cline{8-13}

  \multicolumn{3}{c|}{\small\sc\textbf{Max Hp}} &
  \multicolumn{3}{c}{\small\sc\textbf{Gold}} &
  \multicolumn{1}{c}{} &
  \multicolumn{6}{Z}{\emph{}}
  \\ \cline{8-13}

  \multicolumn{3}{c|}{\LARGE{}} &
  \multicolumn{3}{c}{\LARGE{}} &
  \multicolumn{1}{c}{} &
  \multicolumn{6}{Z}{\emph{}}
  \\ \cline{1-6} \cline{8-13}
  &  &  &  &  &  &  &  &  &  &  &  & \\
\end{tabularx}

% }}}

% {{{ campaign

\pagebreak
\chapter{Bad Water}
\begin{multicols}{2}

\subsection{Hook}
tell players that they can pick up rocks from the ground, get a stick, "borrow" a shovel, etc. that you can ask if there's a boulder or a stick or anything else lying around

the party hears a rumour about a small remote village that became known for its ale and decides to check it out

on the way there they meet a travelling merchant, a dark elf named ginchi. he sparks the conversation

\begin{boxed}
Во время совместных путешествий до вас донеслись слухи о небольшой деревне, которая недавно прославилась своим элем. Заинтригованные проспектами диковенной выпивки, вы в пути уже несколько дней. И вот, наконец, деревья расступаются и вы выходите на дорогу через поле.

Утро, погода стоит ясная, на небе нет ни облачка. По сторонам дороги расстилаются золотые солодовые поля. Люди, двигаясь рядом, одновременно взмахивают косами, но песни не слышно.

Вас догоняет повозка запряженная лошадью, судя по всему торговца разными мелочами. За упряжкой-- темный эльф, в зубах у него травинка, а на голове соломенная шляпа.

"Тоже хотите изведать здешней выпивки?"
\end{boxed}

ghinci's dialogue ideas:

\begin{boxed}
Я -- Гинчи, а лошадь зовут Пташка

Раньше был наемником, но мне отяготела рисковая жизнь.

\textbf{cha low}. Друг погиб на одном из заданий, мы пытались стянуть семейную реликвию, какой-то там меч. Стрела меж глаз. Я не смог ничего сделать.

Теперь я странствующий торговец
\end{boxed}

\section{Act I}
\subsection{Village}
two buildings, a tavern and a brewery. 

\begin{boxed}
Вы вместе прибываете в деревню. Картина перед вами скромная: несколько домов сложенных из бревен, а среди них только пивоварня и таверна.

"Взгляните на мои товары?"\,, говорит Гинчи, "скидки новым покупателям" *Гинчи показывает на свою повозку*

\end{boxed}

give the players a list of things they can buy. 

each of the players can get one item off of Ghinci for half the price.

\begin{boxed}
Я пойду выпить, вы со мной?
\end{boxed}

\subsection{Tavern}
the tavern is full of people, the usual. BUT! nobody talks about anything that happened soon-ish, as they all are forgetting that

\begin{boxed}
На вывеске таверны название -- "Коготь Медведя". Изнутри громкие разговоры и редкие раскаты смеха.

помещение хорошо освещено, стены бревенчатые. несколько штук круглых столов, барная стойка. на стене голова медведя. как бы убрано, но не сказать чтобы очень чисто

вы слышите голос

"Присаживайтесь, присаживайтесь мои хорошие!"

к вам направляется хозяин таверны, добродушно улыбающийся дворф, в руках у него метла, и он судя по всему только что подметал пол.

"Чего желаете?"
\end{boxed}

give players a print out menu of drinks with one highlight especially clearly (being that rumoured ale). if the players for some reason don't order it, serve it to them anyway. fore-fucking-shadowing

ask the players if they want to say toasts

\begin{boxed}
  сказать, что выпивка хороша, значит ничего не сказать.

  \textbf{per low} но есть странный привкус. Такой сладкий, но земельный, что ли. 
\end{boxed}

have ghinci order two more drinks and down them quickly after a toast

have players drink, if they interact with the tavern keeper have him forget them

if they walk out have Ghinci forget what happened

\begin{boxed}
  Ухх, хорошо выпили, давной я так ни с кем не сидел. Подождите... а вы вообще кто?
%  \textbf{per low} С виду Гинчи выглядит странновато, как будто в ауте что ли. Может это из-за алкоголя. 
%  Гинчи: я... как я сюда попал? вы вообще кто? (погоди... тебя я помню *указывает на темного эльфа если в пати такой есть*)
\end{boxed}

if they go back

\begin{boxed}
  Хозяин таверны вернулся за стойку. \textbf{per low} На ней стоит почти выпитая кружка эля.

  "О, посетители? Чего желаете? Комнату? Выпить?"

  ... Да обычная выпивка, я ей уже сколько лет торгую с тех пор как у меня родилась дочь... как же ее звали?

  Оставьте меня, мне нужно побыть одному
\end{boxed}

if the players rent a room, have time pass weirdly, so that its the twilight again

\subsection{Brewery}
there's a black cat called Juro.

if they go here first

\begin{boxed}
  составлена из бревен, внутри несколько человек. один перебирает зерно, двое других тоскают мешки, еще один размешивает что-то в чане. вокруг стоят уже закрытые бочки
\end{boxed}

if they go here after the tavern:

\begin{boxed}
  Время для всех пролетело незаметно, уже вечереет. Работа на сегодня окончена, и пивоварня закрыта наглухо.

  \textbf{per mid} в траве свернувшись в клубочек лежит черный кот. в сумерках его почти не видно. его глаза поблескивают и направлены на вас

  ... Для остальных это звучит как мяуканье. 

  Чувак, ты че замяукал вдруг? У кота на морде вырисовывается крайнее изумление.

  Чувак, тут что-то неладное, люди ведут себя как-то странно, а по ночам я вижу как вокруг колодца ошиваются странные типы. А эти (показывает лапой на пивоварню) потом на этой воде выпивку варят.
\end{boxed}

\subsection{Well}
\begin{boxed}
  По дороге у вас ногами у вас шуршит трава

  Перед вам обычный деревенский колодец, с виду ничего необычного. Глубокий, внизу вода.

  Видно, что им часто пользуются, но веревка выглядит надежно

  \textbf{per low} тот же странный сладковато-земельный запах
\end{boxed}

\section{Act II}
\subsection{Descent}
by now the party should be hooked and swarm the well. have them dive into water one way or another (roll endurance checks for suffocation, maybe make it just a bit dramatic)

the party emerges in the beginning of what looks like an old ruin/dungeon. the place is pretty humid, moss everywhere, rats squeaking somewhere, very dark.

glowing and regular mushrooms, the regular ones are piosonous

\subsection{Dungeon}
pack a bunch of e. a. poe into this. same kind of gothic unsettling horror with a tinge of madness (think cask of amontillado and the pit and the pendulum)

the layout should be separated into floors (4+boss arena?) that get more and more dangerous and typical fantasy-ish. things progress from just rats, to animated skeletons, to traps, to cultists.

room/encounter ideas:
\begin{enumerate}
  \item a door that players need to push together to open
  \item some stupid animated thing like a broom or something that fights them
  \item a trap room that fills with water (yes, that water) to the ceiling in a few turns
\end{enumerate}

\subsection{Interrogation}
around half way through the players may capture a cultist and interrogate him. tell them some stuff about the final boss (section below).

\section{Act III}
\subsection{Final Boss}
make it some large skeleton with glow in the dark bones. this ancient dude is some kind of a demigod that the cultists worship.

have it be a gimicky-ish bossfight where becomes visible/attackable for a turn after you shine light onto him, and then fades away and becomes invisible/invincible. also he's invisible if there's any light around.

when the party enders the room have the lights on, once the combat starts he breaks the torches and they see the green glow, next turn he fades.

let's say that he hasn't regained his physical form yet, and he needs souls to get it back and that's his motive. also he likes getting drunk? scratch that, makes him too likeable.

\subsection{Reveal}
the cultists perform rituals that poison the water in the well, making the water take VERY GOOD and take the souls of those who drink it. the souls are then fed into skeleton and make him more powerful

% }}}

\end{multicols}
\end{document}
