\documentclass[a5paper,11pt]{book}
% {{{
\usepackage[utf8]{inputenc}
\usepackage[russian]{babel}
\usepackage[document]{ragged2e}

\usepackage{indentfirst}
\usepackage{fancyhdr} % page numbers
\usepackage{titlesec} % titles customization
\usepackage{enumitem}
\usepackage{multicol}
\usepackage{framed}
\usepackage{CormorantGaramond}
\usepackage[margin=0.6in, bottom=0.7in]{geometry}

% titles
\titlespacing*{\chapter}{0pt}{12pt plus 4pt minus 4pt}{8pt plus 1pt minus 1pt}
\titlespacing*{\section}{0pt}{12pt plus 4pt minus 4pt}{10pt plus 1pt minus 1pt}
\titlespacing*{\subsection}{0pt}{8pt plus 2pt minus 2pt}{4pt plus 0pt minus 0pt}

\titleclass{\chapter}{straight}

%\definecolor{reddish}{RGB}{76,28,12}
\titleformat{\chapter}
  {\huge\bfseries\scshape} % format
  {}                % label
  {0pt}             % sep
  {} % before-code
  [{\titlerule[0.1pt]}]

\titleformat{\section}
  {\large\bfseries\scshape} % format
  {}                % label
  {0pt}             % sep
  {\Large}          % before-code
  [{\titlerule[0.1pt]}]

\titleformat{\subsection}
  {\large\bfseries\scshape} % format
  {}                % label
  {0pt}             % sep
  {\large}          % before-code

% set page numbers at the bottom
\pagestyle{fancy}
\fancyhf{}
\renewcommand{\headrulewidth}{0pt}
\cfoot{\thepage}

% remove hyphenation
\hyphenpenalty=10000

\setlength{\parindent}{2em}
\setlength{\parskip}{0.8em}

\setlist[itemize]{itemsep=0.0em, topsep=0pt, leftmargin=1em}

% }}}

\begin{document}
\chapter{Горе Кузнеца}
%\begin{multicols}{2}
Гоблины похищают дочь кузнеца. Они хотят принести ее в жертву в гробнице рыцарей, чтобы превратить ее в святилище своего бога.
\section{Tavern}

\begin{framed}
На равнину спустились сумерки. После долгих поисков тепла, ночлега и чем хорошенько набить желудок, трое искателей приключений находят себя в таверне близ Кобаньего Леса.

Внутри, за ужином несколько крестьян обсуждают события прошедшего дня. В их разговоре мало примечательного-- каждый новый день в долине похож на предыдущий. К одной из стен прибит пергамент.
\end{framed}

\begin{itemize}
  \item За столом: подбегает паренек, спрашивает что будут есть. Распечатка меню.
  \item К одной из стен гвоздем пребит пергамент.
  \item Perception: Один крестьянинов низким голосом упомянул гоблинов.
  \item Мета знакомство во время еды.
\end{itemize}

\begin{framed}
... В таверну врывается грузный, бородатый мужчина в черном кожанном фартуке. От него пахнет серой, а в его руках тяжелый молот.

-- Они забрали Бесс. Они забрали мою дочь.
\end{framed}

\begin{itemize}
  \item Кто? Гоблины. Это были они. Я слышал крики на их языке.
  \item Что в лесу? Кабаны. Мы там охотимся.
  \item Мини-детективность: следы уходящие в лес.
\end{itemize}

\section{Forest}
Лес слишком большой что-бы пересечь его за пару часов. Игроки устанут и разобьют лагерь. РП? Вкинуть пару кобанов посреди ночи.

\section{Outside}
Через час-два пути после леса, игроки увидят древнее каменное сооружение в склоне холма. Спросить как они ведут себя, скрытно или нет.

У ворот стоят два гоблина, облокотившись на оружие, зевают.

Округу патрулируют два гоблина.
\begin{itemize}
  \item Если игроки подождут и спрячутся, то смогут заметить патруль. Stealth vs Perception checks. Как только первый гоблин будет ранен, они начнут отступать/сражаться 1-1. Если добегут до входа, то на гоблинском заорут что приближается опасность. Два стража у ворот гробницы уйдут внутрь.
  \item Если игроки войдут сразу, патруль застанет их в 1й комнате. Те что у ворот тут же уйдут внутрь (они знают что им не победить).
\end{itemize}

\section{Entrance}

Толстые деревянные двери гробницы были запечатаны. Сейчас они проломлены и кругом лежат щепки. Лестница вниз. Внутри кромешная тьма. Вонь, гниль, по всюду грязь.

Если прислушаться, то можно услышать чант доносящийся глубоко изнутри.

\section{Brazier Room}

\begin{itemize}
  \item Неколько гоблинов 3-5. В зависимости от того что со стражами и патрулем
  \item Если патруль не убит, то они зайдут сзади.
\end{itemize}

По середине чаша в которой разжигали огонь. У стены алтарь для подношений. Мб какой-то лут (зелье или свиток) который не заметили гоблины.

На стене вырезаны образы рыцарей, дающих клятву, спасающих людей из огня, сражающих страшного зверя. На стене напротив текст клятвы рыцарей:

\begin{framed}
Я, Эндир Валдон, клянусь служить ордену Делиана. Клянусь служить порядку и бороться с хаосом. Клянусь унести секреты ордена с собой в могилу
\end{framed}

\section{Trap}
Плита на полу нажимается только тяжелыми персонажами (не гоблинами). Маятник с лезвием поперек коридора. Урон-- импров, по ситуации.

Может быть хорошим временем для короткого отдыха. Без опасности.

Из-за поворота слышен чант шамана и сдавленный женский плач.

\section{Ritual Room}
Дверь сюда призакрыта.

Запах жженой инценции. Ближе к дальней стене статуя. Перед ней алтарь, шаман читает заклинание склонившись девушкой, связанной по рукам и ногам. Ее рот заткнут трапкой, одежда на ней изорвана, она вся в грязи, рыдает и пытается вырваться.

\begin{itemize}
  \item Шаман. Гоблин с клерик спеллом и кинжалом.
  \item Багбер. Если кто-то из группы упадет, он на ломанном человечьем скажет, "уходи-те и ваш друг выживет". Предлагая продать жизнь упавшего в обмен за жизнь девчонки.
\end{itemize}

\begin{framed}
Надпись на статуе: "Если тебе суждено это сдержать, то сначала ты должен отдать это мне"
\end{framed}

Потайная дверь откроется если сказать статуе "слово" или "клятва" или произнести клятву.

В комнате затхлый запах, очень много пыли. Несколько гробов.

\begin{itemize}
  \item Пара зомби которые атакуют если потревожены.
\end{itemize}

73G, зелье

\clearpage

\begin{framed}
\begin{center}
НАЗНАЧЕНА НАГРАДА!

ИЗБАВЬ ЭТИ ЗЕМЛИ ОТ НЕЧИСТИ!
\begin{itemize}[leftmargin=9em]
  \item 5S за ухо кобольда!
  \item 20S за пару гоблинских ушей!
  \item 10G за голову багбера!
  \item 40G за клык огра!
\end{itemize}
\end{center}
\end{framed}
%\end{multicols}
\end{document}


