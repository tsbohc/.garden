\documentclass[a4paper,11pt]{book}

    \usepackage{lipsum}

    % fonts and languages
    \usepackage{fontspec}
    \usepackage[utf8]{inputenc}
    \usepackage[russian]{babel}
    \usepackage{Alegreya}
    % general document formatting
    \usepackage[margin=1.4cm]{geometry}
    \usepackage{titlesec} % titles customization
    % columns
    \usepackage{multicol}
    \setlength\columnsep{0.7cm}
    % indent first paragraph
    %\usepackage{indentfirst}


    % good tables
    \usepackage{tabularx}

    \newcolumntype{L}{>{\raggedright\arraybackslash}X}
    \newcolumntype{R}{>{\raggedleft\arraybackslash}X}
    \newcolumntype{C}{>{\centering\arraybackslash}X}

    % alternate rows on tables
    \usepackage[table]{xcolor}

    \definecolor{lightgray}{HTML}{f5f5f5}
    \let\oldtabularx\tabularx
    \let\endoldtabularx\endtabularx
    \renewenvironment{tabularx}{
        \bigskip
        \noindent
        \rowcolors{2}{}{lightgray}
        \oldtabularx
        } {
        \endoldtabularx
        \medskip
        }

\usepackage{graphicx}
\usepackage{eso-pic}



    % boxed text
    \usepackage[most]{tcolorbox}
    \usepackage{tikz}

    \newenvironment{Frame}{%
        \begin{tcolorbox}[%
            notitle, sharp corners, colback=white!98!black,
            frame hidden,
            borderline west = {0.1pt}{0pt}{black},
            borderline east = {0.1pt}{0pt}{black},
            %borderline north = {0.5pt}{0pt}{black!70!white},
            %borderline south = {0.5pt}{0pt}{black!70!white},
            boxrule=0.5pt, boxsep=0pt, enhanced,
            %shadow={0pt}{0pt}{-2pt}{opacity=0.1,black}
            fuzzy shadow={0pt}{0pt}{-1pt}{0.6pt}{opacity=0.04,white!70!black}
        ]%
        \setlength{\leftskip}{0.7em}
        \setlength{\parindent}{-\leftskip}
    }{%
        \end{tcolorbox}
    }

    %\newenvironment{Frame}[1][]{%
    %    \begin{mdframed}[%
    %        linecolor=gray,
    %        linewidth=0.5pt,
    %        backgroundcolor=white,
    %        topline=false,
    %        bottomline=false,
    %        shadow={3pt}{-3pt}{0pt}{opacity=1,black}
    %    ]%
    %}{%
    %    \end{mdframed}
    %}


    % wider rows in tables
    \renewcommand{\arraystretch}{1.2}

\usepackage{enumitem}
\setlist[itemize]{itemsep=-0.2em, topsep=0em, leftmargin=1em}

% titles
\titlespacing*{\chapter}{0pt}{12pt plus 4pt minus 4pt}{8pt plus 1pt minus 1pt}
\titlespacing*{\section}{0pt}{12pt plus 4pt minus 4pt}{10pt plus 1pt minus 1pt}
\titlespacing*{\subsection}{0pt}{8pt plus 2pt minus 2pt}{4pt plus 0pt minus 0pt}

\titleclass{\chapter}{straight}

\titleformat{\chapter}
  {\huge\bfseries\scshape} % format
  {}                % label
  {0pt}             % sep
  {}                % before-code
  [{\titlerule[0.1pt]}]

\titleformat{\section}
  {\large\bfseries\scshape} % format
  {}                % label
  {0pt}             % sep
  {\Large}          % before-code
  [{\titlerule[0.1pt]}]

\titleformat{\subsection}
  {\large\bfseries\scshape} % format
  {}                % label
  {0pt}             % sep
  {\large}          % before-code

% remove hyphenation
\hyphenpenalty=8000

\begin{document}

\AddToShipoutPictureBG{\includegraphics[height=\paperheight]{bg.png}}

\begin{multicols}{2}

\chapter{Ruleset}

D\&D -- не видеоигра. Взаимодействие с миром здесь приближено к реальному. Например:

\begin{Frame}
\textbf{DM}: Замок старый и весь проржавел, тебе придется постараться.

\textbf{TH}: А что если я использую масло от своей лампы?

\textbf{DM}: Хорошая идеа! Можешь бросать с преимуществом.
\end{Frame}

%Используйте свое окружение, задавайте конкретизирующие воспросы и подходите к решению проблем со смекалкой.

Многие обыденные предметы (особенно те, что у вас в рюкзаке) могут значительно повлиять на успех того или иного действия.

\subsection{Combat}

В бою, не думайте о мече как о наборе чисел который заставляет здоровье врагов уменьшаться.

Выбирайте куда целиться, и какой удар нанести: режущий, колящий или вообще рукоятью:

\begin{Frame}
\textbf{WA}: Я перехватываю меч за лезвие и наношу удар крестовиной, целясь ему в голову... 20!

\textbf{DM}: Ты попадаешь в его шлем и раздается звонкий гул. Оглушенный, он теряет равновесие и падает на колени.
\end{Frame}

Продуманные действия могут дать преимущество и способствуют более динамичному, интересному (и кровавому) бою!

\section{Alignment}

За моим столом, это способ объяснить свои действия, а не их характеризация. Плохие герои могут совершать хорошие поступки, пускай и из своих корыстных побуждений.

Рассмотрим пример с Робин Гудом:

\begin{itemize}
    \item \textbf{good}
    \begin{itemize}
        \item \textbf{lawful}: Я верен истинному королю, поэтому помогаю его людям и противостою ложному королю.
        \item \textbf{chaotic}: Я ворую золото у богатых и отдаю его бедным, потому что они более несчастны.
    \end{itemize}
    \item \textbf{evil}
    \begin{itemize}
        \item \textbf{lawful}: Я хочу отомстить нынешнему королю, и верен коду разбойников.
        \item \textbf{chaotic}: Я отдаю золото бедным чтобы усугубить отношения между королем и людьми, что приведет к народному восстанию.
    \end{itemize}
\end{itemize}


\section{Critical Damage}

\chapter{Sensible Weather}

Frankly, weather has been neglected in the official D\&D ruleset. This is an attempt to add some meaningful decision making, without making it too much of a hassle for the players.

\begin{itemize}
    \item \textbf{Precipitation}: if resting without cover, you must succeed constitution save (DC 12 or 16) to gain the benefits of a long rest.
    \item \textbf{Snow}: advantage on tracking footprints.
    \item \textbf{Rain}: disadvatage on checks related to climding or keeping balance.
\end{itemize}

\section{Weather Conditions}

To determine the starting weather conditions, roll 1d8 with a -1 or -2 modifier in cold climate, or a +1 or +2 modifier in hot climate:

\begin{tabularx}{\linewidth}{ c l c X }
    \textbf{d8} & \textbf{condition} & \textbf{wind} & \textbf{temperature} \\
   -1 & blizzard      & +2 & severely cold \\
    0 & snow          & +1 & very cold     \\
    1 & thunderstorm  & +1 & cold          \\
    2 & rain          & +1 & cold          \\
    3 & cold front    & +0 & cool          \\
    4 & overcast      & +0 & cool          \\
    5 & cloudy        & +0 & warm          \\
    6 & partly cloudy & +0 & warm          \\
    7 & clear sky     & -1 & hot           \\
    8 & warm front    & -1 & hot           \\
    9 & draught       & -1 & very hot      \\
   10 & heatwave      & -2 & severely hot  \\
\end{tabularx}

Then, roll 1d4 to determine wind speed, with its modifier:

\begin{tabularx}{11em}[t]{ c X }
    \textbf{d4} & \textbf{wind speed} \\
   -1 & none      \\
    0 & none          \\
    1 & light  \\
    2 & weak          \\
    3 & medium    \\
    4 & strong      \\
    5 & very strong        \\
    6 & extremely strong \\
\end{tabularx}
\begin{tabularx}{11em}[t]{ c X }
    \textbf{d8} & \textbf{change} \\
    1 & down on the list  \\
    2 & more precipitation    \\
    3 & stronger wind      \\
    4 & similar        \\
    5 & reroll \\
    6 & weaker wind     \\
    7 & less precipitation    \\
    8 & up on the list      \\
\end{tabularx}

Whenever appropriate (e.g once a day), roll 1d8 on the weather change table to determine the next weather condition.


\bigskip
\lipsum

\end{multicols}
\end{document}
