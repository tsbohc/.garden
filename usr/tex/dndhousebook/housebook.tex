\documentclass[a4paper,11pt]{book}

    \usepackage{lipsum}

    % fonts and languages
    \usepackage{fontspec}
    \usepackage[utf8]{inputenc}
    \usepackage[russian]{babel}
    \usepackage{AlegreyaSans}
    \usepackage{Alegreya}
    %\setmainfont{QTFraktur}

    % general document formatting
    \usepackage[margin=1.4cm]{geometry}
    \usepackage{titlesec} % titles customization
    % columns
    \usepackage{multicol}
    \setlength\columnsep{0.7cm}
    % indent first paragraph
    %\usepackage{indentfirst}

    % table of contents and page numbering
    \usepackage{tocloft}
    \renewcommand*{\cftsecindent}{2em}
    \renewcommand*{\cftsubsecindent}{3em}
    \setcounter{secnumdepth}{0}
    \pagestyle{plain}

    % good tables
    \usepackage{tabularx}

    \newcolumntype{L}{>{\raggedright\arraybackslash}X}
    \newcolumntype{R}{>{\raggedleft\arraybackslash}X}
    \newcolumntype{C}{>{\centering\arraybackslash}X}

    % alternate rows on tables
    \usepackage[table]{xcolor}

    \definecolor{lightgray}{HTML}{fafafa}
    \let\oldtabularx\tabularx
    \let\endoldtabularx\endtabularx
    \renewenvironment{tabularx}{
        \AlegreyaSansOsF
        \bigskip
        \noindent
        \rowcolors{2}{}{white!94!blue}
        \oldtabularx
        } {
        \endoldtabularx
        \medskip
        }


    % background
    \AtBeginDocument{
        \AddToShipoutPictureBG{\includegraphics[height=\paperheight]{bg.png}}
    }

    % italic command
    \newcommand{\ii}[1]
    {{\AlegreyaSansOsF
      \itshape
      #1}}{}

    \newcommand{\sans}[1]
    {{\AlegreyaSansOsF#1}}

    % highlight box
    \AtBeginDocument{
        \newtcbox{hlbox}[1][]{
            enhanced,colback=white!83!blue,
            interior style={opacity=0.4},
            frame hidden,
            nobeforeafter,tcbox raise base,shrink tight,extrude by=1mm
        }
    }

    \usepackage{tikz}

    % highlight command
    %\newcommand{\hl}[1]{
    %    \hlbox{\AlegreyaSansOsF#1}
    %}

    \newcommand*{\hl}[1]{%
      \tikz[baseline] \node[rectangle, fill=white!92!blue, rounded corners, inner sep=0.3mm, anchor=base]{\AlegreyaSansOsF#1};%
    }

\usepackage{graphicx}
\usepackage{eso-pic}


\colorlet{bordercolor}{black!70!blue}

\setlength{\parindent}{1.5em}

    % boxed text
    \usepackage[most]{tcolorbox}

    \newenvironment{Frame}{%
        \begin{tcolorbox}[%
            drop lifted shadow,
            notitle, sharp corners, colback=white!98!black,
            frame hidden,
            borderline west = {0.1pt}{0pt}{gray!10!bordercolor},
            borderline east = {0.1pt}{0pt}{gray!10!bordercolor},
            borderline north = {1pt}{0pt}{bordercolor},
            borderline south = {1pt}{0pt}{bordercolor},
            boxrule=0.5pt, boxsep=0pt, enhanced,
            %shadow={0pt}{0pt}{-2pt}{opacity=0.1,black}
            fuzzy shadow={0pt}{0pt}{-0.5pt}{0.8pt}{opacity=0.005,white!30!gray}
        ]%
    }{%
        \end{tcolorbox}
    }

    \newenvironment{Conversation}{
        \smallskip
        \begin{tcolorbox}[%
            drop lifted shadow,
            notitle, sharp corners, colback=white!98!black,
            frame hidden,
            borderline west = {1pt}{0pt}{bordercolor},
            borderline east = {1pt}{0pt}{bordercolor},
            borderline north = {0.1pt}{0pt}{gray!10!bordercolor},
            borderline south = {0.1pt}{0pt}{gray!10!bordercolor},
            boxrule=0.5pt, boxsep=0pt, enhanced,
            %shadow={0pt}{0pt}{-2pt}{opacity=0.1,black}
            fuzzy shadow={0pt}{0pt}{-0.5pt}{0.5pt}{opacity=0.005,gray}
        ]%
        \setlength{\leftskip}{0.7em}
        \setlength{\parindent}{-\leftskip}
        \AlegreyaSansOsF
    }{
        \end{tcolorbox}
        \smallskip
    }


    \newenvironment{Item}[2]
    {
     \smallskip
     \begin{Frame}
     \section{#1}
     \ii{#2}
     \smallskip \\
     \setlength{\parindent}{1.5em}
    }{
        \smallskip
        \end{Frame}
    }


    % wider rows in tables
    \renewcommand{\arraystretch}{1.05}

    % list settings
    \usepackage{enumitem}
    \setlist[itemize]{itemsep=-0.2em, topsep=0.5em, leftmargin=1em}
    \setlist[enumerate]{itemsep=-0.2em, topsep=0.5em, leftmargin=1em}

% titles
\titlespacing*{\chapter}{0pt}{12pt plus 4pt minus 4pt}{8pt plus 1pt minus 1pt}
\titlespacing*{\section}{0pt}{12pt plus 4pt minus 4pt}{10pt plus 1pt minus 1pt}
\titlespacing*{\subsection}{0pt}{8pt plus 2pt minus 2pt}{4pt plus 0pt minus 0pt}

\titleclass{\chapter}{straight}

\titleformat{\chapter}
{\Huge\bfseries\scshape}
{}{0em}{}[\titleline{\color{bordercolor}\titlerule[0.3pt]}]

\titleformat{\section}
{\LARGE\bfseries\scshape}
{}{0em}{}[\titleline{\color{bordercolor}\titlerule[0.3pt]}]

\titleformat{\subsection}
    {\large\bfseries\scshape} % format
    {}                % label
    {0pt}             % sep
    {\Large}          % before-code

% remove hyphenation
\hyphenpenalty=8000

\begin{document}
\begin{multicols}{2}

\renewcommand\contentsname{Table of Contents}
\tableofcontents
\newpage

\chapter{Foreword}

D\&D -- не видеоигра. Глубина взаимодействия с миром здесь приближена к реальности:

\begin{Conversation}
\textbf{DM}: Замок старый и весь проржавел, тебе придется постараться.

\textbf{TH}: А что если я использую масло от своей лампы чтобы его немного расшевелить?

\textbf{DM}: Хорошая идея-- можешь бросать с преимуществом.
\end{Conversation}

Многие обыденные предметы (особенно те, что у вас в рюкзаке) могут значительно повлиять на исход ситуации.

\subsection{Combat}

В бою, не думайте о мече как о наборе чисел который заставляет здоровье врагов уменьшаться.

Выбирайте куда целиться, и какой удар нанести: режущий, колящий или вообще рукоятью:

\begin{Conversation}
\textbf{WA}: Я перехватываю меч за лезвие и наношу удар крестовиной, целясь ему в голову... 20!

\textbf{DM}: Ты попадаешь в его шлем и раздается звонкий *тунк*. Оглушенный, он теряет равновесие и падает на колени.
\end{Conversation}

То же самое применимо как к стрелковому оружию, так и к магии.

Продуманные действия дают преимущество над врагом и способствуют более динамичному, интересному (и кровавому) бою!

\subsection{Rule of Cool}

DM готов проигнорировать здравый смысл, если это способствует крутому моменту в игре. Иногда.

\chapter{House Rules}

Правила, приведенные ниже, существенно меняют некоторые механики игры.

Прочитайте эту главу внимательно.

\section{Ability Scores}

Для определения характеристик на этапе создания персонажа используется следующий метод:

\begin{enumerate}
    \item Бросаем 3d10, записываем наибольший ролл, повторяем 6 раз.
    \item Распределяем [6 5 4 3 2 1] по 1 значению на каждый ролл из предыдущего шага.
    \item Свободно присваиваем полученные значения характеристикам.
\end{enumerate}

%\subsection{Analysis}
%
%Total scores average at 68.85, ranging from 59 to 78 within >1\% probability. This is an overall power decrease of 6.28\% compared to 4d6 drop the lowest, which averages at 73.47, ranging from 61 to 86 within >1\% probability. Also, the variablity between characters is reduced by 4 points, ensuring more consistent groups.
%
%%Estimated distribution:
%%
%%\begin{tabularx}{\linewidth}{ c C C C C C C }
%%    \textbf{a.s} & \textbf{value} & \textbf{\%} & \textbf{value} & \textbf{\%} & \textbf{value} & \textbf{\%} \\
%%    1. & 16 & 85\% & 15 & 13\% & 14 & 2\%  \\
%%    2. & 15 & 52\% & 14 & 36\% & 13 & 10\% \\
%%    3. & 13 & 43\% & 12 & 26\% & 14 & 20\% \\
%%    4. & 11 & 34\% & 12 & 27\% & 10 & 22\% \\
%%    5. &  9 & 28\% & 10 & 23\% &  8 & 22\% \\
%%    6. &  6 & 22\% &  7 & 22\% &  5 & 18\% \\
%%\end{tabularx}
%
%Rolls are capped at 16 to accomodate for racials without introducing stat caps. Hitting 85\% with a +2 racial allows one to hit 20 in a score on level 4. On the other hand, flaws are more prevalent, ranging from 2-3 on average.
%
%Such distribution should help separate players by areas of expertise and let each one have their time under the spotlight.

\section{Alignment}

За моим столом, это способ объяснить свои действия, а не их характеризация. Плохие герои могут совершать хорошие поступки, пускай и из своих корыстных побуждений.

Рассмотрим это на примере Робин Гуда:

\begin{itemize}
    \item \textbf{Lawful Good}: Я верен истинному королю, поэтому помогаю его людям.
    \item \textbf{Chaotic Good}: Я ворую золото у богатых и отдаю его бедным, потому что они нуждаются в нем больше.
    \item \textbf{Lawful Evil}: Я хочу отомстить нынешнему королю, и верен коду разбойников.
    \item \textbf{Chaotic Evil}: Я краду золото у богатых чтобы усугубить их отношения с королем, что приведет к народному восстанию.
\end{itemize}

\begin{Item}{Bottled Vitum}{Бутылка Витума -- Волшебный предмет}
    Фляга из матового, тускло-зеленого стекла, полная бодрящего напитка. Максимальное число зарядов равно уровню персонажа.

    \textbf{Забвение.} Во время отдыха, вы можете потратить один или более зарядов чтобы восстановить hd + мoд. выносливости (мин. 1) здоровья за каждый.

    \textbf{Абстиненция.} После продолжительного отдыха пополняется на 1/2 уровень персонажа зарядов. Продолжительный отдых не восстанавливает здоровье.

    \begin{tabularx}{\linewidth}{ c L }
        \textbf{hd} & \textbf{Классы} \\
        d6  & Маг, Чародей \\
        d8  & Бард, Жрец, Друид, Колдун, Монах, Плут, Изобретатель \\
        d10 & Воин, Паладин, Следопыт \\
        d12 & Варвар
    \end{tabularx}
\end{Item}

\section{Skill Trials}

Иногда, для достижения определенной цели группе необходимо решить несколько задач.

В таких случаях, DM устраивает \ii{испытание}-- серию проверок характеристик в которых участвует вся группа.

%Например, для успешного побега из темницы необходимо:
%
%\begin{itemize}
%    \item \textbf{Выбраться из камеры}: достав ключи, взломав замок или использовав магию?
%    \item \textbf{Замести следы}: спрятать тело? смыть следы крови? запутать преследователей?
%    \item \textbf{Сбежать незамеченным}: пуститься прочь сломя голову или медленно красться через тени?
%\end{itemize}
%
%Специализация в убеждении даст преимущество при попытке обмануть стража, ловкость рук-- во взломе замка, etc.
%
%У каждого героя будет свой подход. В их интересах-- играть на стороне своих сильных сторон.

\section{Misc}

\subsection{Rest}
Продолжительный отдых не восстанавливает здоровье. Во время отдыха можно использовать \textit{Bottled Vitum}

\subsection{Critical Damage}
При выпадении нат. 20, урон равен сумме броска урона и его максимальному урону.

На примере двуручного меча: 2d6 + 12.


\chapter{Magic Items}
Индекс магических предметов сотворенных искренне вашим DM.

\begin{Item}{Stone of Detection}{Камень Определения -- Волшебный предмет}
    Этот обычный на вид камень сферической формы скрывает в себе большой потенциал в области определения.

    \textbf{Гравитация.} Выпустите камень из рук. Он упадет, определяя направление и интенсивность гравитации.

    \textbf{Наклон.} Положите камень на плоскую поверхность. Он покатится, определяя направление и степень наклона поверхности. Может не сработать на мягких поверхностях.

    \textbf{Иллюзия.} Вы можете бросить камень на расстояние до 30ft. Он укажет вам на иллюзию если пройдет сквозь существо или предмет.

    \textbf{Невидимость.} Бросьте камень. В случае если его траектория резко прервется, это укажет вам на невидимое существо или предмет.

    Способности этого камня вселяют страх в магов и волшебников, не способных почувствовать в нем хоть толику магии.
\end{Item}

Практическая шутка. Но кому меня винить?

\begin{Item}{Staff of Nustered}{Посох Нустэрд -- Зачарованный посох}
    Деревянный посох, на конце увенчанный крыглым туманным кристаллом, бережно обхваченным ответвлениями.

    При попытке использования, посох обнаруживает магию в неком твердом сферическом объекте совсем рядом. Кристалл испускает мягкий теплый свет в радиусе 10ft.

    \textbf{Nust.} Если к посоху отнесутся с пренебрежением, мягкий свет прервется яркой вспышкой, наносящей держащему 1 урона.

    \textbf{Ered.} Если к посоху отнесутся хорошо, он будет раз в день даровать владельцу +1 в проверках Арканы.
\end{Item}

Можно спрятать в ветвях дерева, и сделать так, что в ночном лесу слабо освещена одна из крон.

\begin{Item}{Mentha Gracilis}{Ментха Грацилис -- Растение, редкое}
    Растение, с виду напоминающее мяту. Листья обжигают холодом при прикосновении. Обладает невероятно свежим ароматом. Имеет 1d4 (мин. 2) листьев.

    \textbf{Ледяное Дыхание.} Листок, употребленный в сыром виде, позволяет вам использовать Ледяное Дыхание: 1d6 урона, конус 10ft.

    \textbf{Ингредиент.} Каждого листка хватает на одно зелье, что повышает урон до 2d4. Вероятно, существуют и другие рецепты.
\end{Item}

Там, где растет Mentha Gracilis, температура ощутимо ниже. По кому-то может пробежать дрожь.

Для приготовления зелья можно использовать испытание. Другие рецепты включают яд и т.д.

\begin{Item}{Lockpick of Secrets}{Отмычка Тайн -- Зачарованный инструмент}
\end{Item}


\bigskip
\lipsum

\chapter{Sensible Weather}

Frankly, weather has been neglected in the official D\&D ruleset. This is an attempt to add some meaningful decision making, without it being too much of a hassle for the players.

\begin{itemize}
    \item \textbf{Precipitation}: if resting without cover, you must succeed constitution save (DC 12 or 16) to gain the benefits of a long rest.
    \item \textbf{Snow}: advantage on tracking footprints.
    \item \textbf{Rain}: disadvatage on checks related to climding or keeping balance.
\end{itemize}

\section{Weather Conditions}

To determine the starting weather conditions, roll 1d8 with a -1 or -2 modifier in cold climate, or a +1 or +2 modifier in hot climate:

\begin{tabularx}{\linewidth}{ c l c X }
    \textbf{d8} & \textbf{condition} & \textbf{wind} & \textbf{temperature} \\
   -1 & blizzard      & +2 & severely cold \\
    0 & snow          & +1 & very cold     \\
    1 & thunderstorm  & +1 & cold          \\
    2 & rain          & +1 & cold          \\
    3 & cold front    & +0 & cool          \\
    4 & overcast      & +0 & cool          \\
    5 & cloudy        & +0 & warm          \\
    6 & partly cloudy & +0 & warm          \\
    7 & clear sky     & -1 & hot           \\
    8 & warm front    & -1 & hot           \\
    9 & draught       & -1 & very hot      \\
   10 & heatwave      & -2 & severely hot  \\
\end{tabularx}

Then, roll 1d4 to determine wind speed, with its modifier:

\begin{tabularx}{11em}[t]{ c X }
    \textbf{d4} & \textbf{wind speed} \\
   -1 & none      \\
    0 & none          \\
    1 & light  \\
    2 & weak          \\
    3 & medium    \\
    4 & strong      \\
    5 & very strong        \\
    6 & extremely strong \\
\end{tabularx}
\begin{tabularx}{11em}[t]{ c X }
    \textbf{d8} & \textbf{change} \\
    1 & down on the list  \\
    2 & more precipitation    \\
    3 & stronger wind      \\
    4 & similar        \\
    5 & reroll \\
    6 & weaker wind     \\
    7 & less precipitation    \\
    8 & up on the list      \\
\end{tabularx}

Whenever appropriate (e.g once a day), roll 1d8 on the weather change table to determine the next weather condition.

%\begin{Item}{Bömfur's Blank}{Волшебный предмет, редкий}
%    Заготовка Бомфура пропитана духом кузнечного мастерства.
%
%    При успешной обработке и переплаве, заготовка навсегда принимает форму любого оружия, сохраняя при этом степень своей зачарованности.
%\end{Item}


\end{multicols}
\end{document}
