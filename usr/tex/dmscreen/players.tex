\documentclass[a4paper,landscape,12pt]{book}

% {{{

\usepackage{fontspec}
\usepackage[utf8]{inputenc}
\usepackage[russian]{babel}
\usepackage[document]{ragged2e}
\usepackage[table]{xcolor}
\usepackage{indentfirst}
\usepackage{multicol}
\usepackage{titlesec} % titles customization
\usepackage{Alegreya}
%\usepackage{CormorantGaramond}
\usepackage[margin=1cm]{geometry}

% no sb
%\renewcommand{\bfdefault}{sb}

%\usepackage{fontspec}
%\newfontfamily\subsubsectionfont[Path=/home/sean/.local/share/fonts]{AlegreyaSC-Medium.ttf}

% titles
\titlespacing*{\chapter}{0pt}{12pt plus 4pt minus 4pt}{8pt plus 1pt minus 1pt}
\titlespacing*{\section}{0pt}{12pt plus 4pt minus 4pt}{10pt plus 1pt minus 1pt}
\titlespacing*{\subsection}{0pt}{8pt plus 2pt minus 2pt}{4pt plus 0pt minus 0pt}

\titleclass{\chapter}{straight}

\titleformat{\chapter}
  {\huge\bfseries\scshape} % format
  {}                % label
  {0pt}             % sep
  {} % before-code
  [{\titlerule[0.1pt]}]

\titleformat{\section}
  {\large\bfseries\scshape} % format
  {}                % label
  {0pt}             % sep
  {\Large}          % before-code
  [{\titlerule[0.1pt]}]

\titleformat{\subsection}
  {\large\bfseries\scshape} % format
  {}                % label
  {0pt}             % sep
  {\large}          % before-code

% remove hyphenation
\hyphenpenalty=10000

\usepackage{enumitem}
\setlist[itemize]{itemsep=-0.2em, topsep=3pt, leftmargin=1.5em}

\newlength\mylength
\setlength\mylength{\dimexpr.5\columnwidth-2\tabcolsep-0.5\arrayrulewidth\relax}
% }}}

\begin{document}
\begin{multicols}{3}
\thispagestyle{empty}

\setlength{\parindent}{0.5em}
\setlength{\parskip}{1pt}

\section{Твой Ход}
За один ход можно:
\begin{itemize}
  \item Переместиться на расстояние, в сумме не превышающее скорость.
  \item Совершить одно \textbf{\textit{действие}}. Движение можно совершить до и/или после действия.
  \item Совершить одно бонусное действие.
  \item Взаимодействовать с одним объектом: открыть дверь, взять предмет со стола...
  \item Произнести короткую фразу.
\end{itemize}

\subsection{Действия}
\begin{itemize}
    \item \textbf{\textit{Атака}} или \textbf{\textit{Заклинание}}.
    \item \textbf{Засада} \textit{(скрытность)}: см. \textbf{\textit{засада}}.
    \item \textbf{Помощь}: дает преимущество другому игроку.
    \item \textbf{Рывок}: удваивает скорость.
    \item \textbf{Отступление}: передвижение не провоцирует атаки возможности.
    \item \textbf{Уклонение}: преимущество на спасброски \textit{ловкости}, помеха на броски атаки по вам.
    \item \textbf{Поиск} \textit{(внимательность} или \textit{анализ)}: обнаружить что-то.
    \item \textbf{Захват} \textit{(атлетика)}: обездвижить существо или оттолкнуть его на 5ft.
    \item \textbf{Лечение} \textit{(медицина)}: стабилизирует существо, +1 hp после 1d4 часов.
    \item \textbf{Импров}: выламывание дверей, обезоруживание, перекаты...
    %\item \textbf{Обезоруживание} \textit{(атака} против \textit{атлетики} или \textit{акробатики)}: выбить оружие из рук противника.
\end{itemize}

\subsection{Инициатива}
\begin{itemize}
    \item \textbf{Ролл}: d20 + \textit{mod ловкости}.
\end{itemize}

\subsection{Атака}
\begin{itemize}
    \item \textbf{Бросок Атаки}: d20 + \textit{mod силы} или \textit{ловкости} + \textit{бонус мастерства}.
    \item \textbf{Бросок Урона}: кость урона + \textit{mod силы} или \textit{ловкости}.
\end{itemize}

\subsection{Заклинание}
\begin{itemize}
    \item \textbf{Сложность Спасброска}: 8 + \textit{mod заклинательности} + \textit{бонус мастерства}.
    \item \textbf{Бросок Атаки}: d20 + \textit{mod заклинательности} + \textit{бонус мастерства}
\end{itemize}

\subsection{Засада}
\begin{itemize}
    \item Вы не можете спрятаться пока на вас смотрят.
    \item Спрятавшись, вы получаете преимущество на атаку против тех, кто вас не видет.
    \item Атака всегда выдает ваше местоположение.
    \item Существа, коротых вы застанете врасплох, пропускают первый ход боя.
\end{itemize}

\section{Отдых}
\textbf{Короткий}, около часа.
\begin{itemize}
    \item Вы восстанавливаете здоровье равное hd + \textit{mod выносливость} за каждый потраченный hd.
\end{itemize}

\textbf{Длинный}, больше 8 часов.
\begin{itemize}
    \item Вы восстанавливаете все потраченное hp и 1/2 max hd.
    \item Вы можете стоять настороже до 2-х часов.
    \item Только один длинный отдых за сутки.
\end{itemize}

\section{Окружение}
\begin{itemize}
    \item \textbf{Полумрак}: помеха при проверках \textit{внимательности}. -5 passive \textit{внимательность}.
    \item \textbf{Тьма}: помеха при атаке, автопровал всех проверок зрения.
    \item \textbf{Укрытие}: 1/2 -- +2 (3/4 -- +5) к AC и спасброскам \textit{ловкости}.
\end{itemize}

\section{Истощение}
Усталость негативно сказывается на бросках атаки, проверках и спасбросках:
\vspace{\baselineskip}

\rowcolors{2}{white}{gray!10}
\begin{tabular}{ c c p{5.8cm} }
    \textbf{1} & 1d4 & -- \\[2pt]
    \textbf{2} & 1d6 & Скорость уменьшается вдвое. \\[2pt]
    \textbf{3} & 1d8 & Только действие или бонусное действие за один ход. \\[2pt]
    \textbf{4} & 1d10 & max hp уменьшается вдвое. \\[2pt]
    \textbf{5} & 1d12 & Скорость падает до 5. \\[2pt]
\end{tabular}

\section{Смерть}
При достижении 0 hp вы теряете сознание.\textsuperscript{1} Восстановите hp чтобы придти в себя.

Если вы начинаете ход с 0 hp, совершите спасбросок смерти (DC 10):

\begin{itemize}
    \item 3 успеха: стабильное состояние, без сознания. 3 провала: смерть.
    \item 1 -- два провала, 20 -- +1 hp.
    \item Получение урона равно провалу (двум за критический урон).
\end{itemize}

\setlength{\parindent}{0pt}
1. Если остаток урона превышает ваше max hp, вы умираете мгновенно.

\end{multicols}
\end{document}
